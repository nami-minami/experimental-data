\documentclass{jlreq}
\usepackage{amsmath,physics2,derivative}
\usephysicsmodule{ab}
\title{Gauss関数の最尤推定}
\author{}

\begin{document}
	\maketitle
	\section*{最尤推定}
		最尤推定

	\section*{Gauss関数の最尤推定}
		正規分布の確率関数はGauss関数であり,以下である.
			\begin{equation}
				f(x;\mu,\sigma^2) = \frac{1}{\sqrt{2\pi\sigma^2}}
				\exp\ab[{-\frac{(x-\mu)^2}{2\sigma^2}}] \label{eq:Gauss}
			\end{equation}
		個数\(N\)のデータ\(x=(x_0,x_1,x_2,\cdots,x_n)\)を用意すると尤度関数は以下と
		なる.
			\begin{equation}
				L(x|\theta) = \sum_{n=0}^N \frac{1}{\sqrt{2\pi\sigma^2}}
				\exp\ab[{-\frac{(x_n-\mu)^2}{2\sigma^2}}] \label{eq:likelihood}
			\end{equation}
		Gauss関数には2つのパラメーターが存在するため,2回に分けて
		各々のパラメーターを決定する必要がある.
		
		\subsection*{パラメーター\(\mu\)についての最尤推定}
			\eqref{eq:likelihood}式を\(\mu\)で偏微分するのだが,
			それに先んじて簡単のため両辺に対数を取って式を変形する.
				\begin{align}
					\log_e L(x|\theta) &= \log_e \sum_{n=0}^N 
						\frac{1}{\sqrt{2\pi\sigma^2}}
						\exp\ab[{-\frac{(x_n-\mu)^2}{2\sigma^2}}]\\
					&= \sum_{n=0}^N\log_e\frac{1}{\sqrt{2\pi\sigma^2}} +
						\log_e e^{-\frac{(x_n-\mu)^2}{2\sigma^2}}\\
					&= \sum_{n=0}^N -\frac{1}{2}\log_e(2\pi\sigma^2) -
					\frac{(x_n-\mu)^2}{2\sigma^2}\\
					&= -\frac{N}{2}\log_e(2\pi\sigma^2) -
						\frac{1}{2\sigma^2}\sum_{n=0}^N(x_n-\mu)^2\label{eq:mu-trans}
				\end{align}
			\eqref{eq:mu-trans}式を\(\sigma\)で偏微分する.
				\begin{equation}
					\pdv{}{\sigma}L(x|\theta) =
						-\frac{1}{\sigma}\sum_{n=0}^N(x_n-\mu)\label{eq:mu-dv}
				\end{equation}
			最後に\eqref{eq:mu-dv}式を0とし,\(\mu\)について解く.
				\begin{align}
					-\frac{1}{\sigma}\sum_{n=0}^N(x_n-\mu) &= 0\\
					-N\mu + \sum_{n=0}^Nx_n &= 0\\
					\mu &= \frac{1}{N}\sum_{n=0}^Nx_n \equiv \bar{x} \label{eq:fish}
				\end{align}
			\eqref{eq:fish}をみるとGauss関数のパラメーター\(\mu\)はデータ\(x\)の
			平均に等しいことがわかる.
			したがって\(\mu\)はデータの平均である.

		\subsection*{パラメーター\(\sigma\)についての最尤推定}
			\(\mu\)の場合と同様にして,\eqref{eq:mu-trans}式を\(\sigma\)で
			偏微分する.対数に注意し,その引数を\(u\)とする.(合成関数の微分)
				\begin{align}
					\pdv{}{\sigma}\log_eL(x|\theta) &= 
						-\frac{N}{2}\odv{}{u}\log_eu\odv{}{\sigma}2\pi\sigma^2 -
						\pdv{}{\sigma}\frac{1}{2\sigma^2}\sum_{n=0}^N(x_n-\mu)^2\\
					&= -\frac{N}{2}\cdot\frac{1}{u}\cdot4\pi\sigma +
						\frac{1}{\sigma^3}\sum_{n=0}^N(x_n-\mu)^2\\
					&= -\frac{N}{2}\cdot\frac{1}{2\pi\sigma^2}\cdot4\pi\sigma + 
						\frac{1}{\sigma^3}\sum_{n=0}^N(x_n-\mu)^2\\
					&= -\frac{N}{\sigma} + 
						\frac{1}{\sigma^3}\sum_{n=0}^N(x_n-\mu)^2\\
					&= \frac{1}{\sigma}\ab\{-N+\frac{1}{\sigma^2}
						\sum_{n=0}^N(x_n-\mu)^2\}\label{eq:sigma-dv}
				\end{align}
			最後に\eqref{eq:sigma-dv}式を0とし,\(\sigma\)について解く.
				\begin{align}
					-\frac{1}{\sigma}\ab\{-N+
						\frac{1}{\sigma^2}\sum_{n=0}^N(x_n-\mu)^2\} &= 0\\
					-N+\frac{1}{\sigma^2}\sum_{n=0}^N(x_n-\mu)^2 &= 0\\
					N\sigma^2 &= \sum_{n=0}^N(x_n-\mu)^2\\
					\sigma^2 &= \frac{1}{N}\sum_{n=0}^N(x_n-\mu)^2 
						\label{eq:sigma-fish}
				\end{align}
			\eqref{eq:sigma-fish}式を見るとこれは分散の定義そのものである.従ってガ
			ウス関数のパラメーター\(\sigma\)はデータの分散である.

			以上2回の最尤推定により,Gauss関数のパラメーターを解析的に決定すること
			ができた.よって,適当なデータを用意し,
			その平均と分散とを\eqref{eq:Gauss}式に代入することでヒストグラムを
			Gauss関数で近似することができる.
\end{document}
