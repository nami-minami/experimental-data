\documentclass{jlreq}
\usepackage{amsmath, physics2, derivative, siunitx}
\title{相対粘度関数の導出}
\author{}

\begin{document}
	\maketitle
	流体が一定の圧力\(p[\si{Pa.s}]\)を受けて内半径\(r[\si{m}]\),
	長さ\(l[\si{m}]\)の毛管内を時間\(\theta[\si{s}]\)要して流れる際,
	流体の体積\(V[m^3]\)には以下のHagen–Poiseuille式が成り立つ.
		\begin{equation}
			V = \frac{\pi p r^4 \theta}{8\mu l}
		\end{equation}
	これを\(\mu\)について解く.
		\begin{equation}
			\mu = \frac{\pi p r^4 \theta}{8Vl}
		\end{equation}
	2流体間の相対粘度即ち粘度比を求める.ここで,流体0には添字0,流体1には添字1を付与
	する.
		\begin{equation}
			\frac{\mu_0}{\mu_1} = 
			\frac{
				\frac{\pi p_0 r_0^4 \theta_0}{8V_0l_0}
			}
			{
				\frac{\pi p_1 r_1^4 \theta_1}{8V_1l_1}
			}
		\end{equation}
	測定に際して,流体0および流体1には同一のOstwald粘度計を用い,体積を揃えたため,
	\(V,r,l\)は同一である.
		\begin{align}
			\frac{\mu_0}{\mu_1}= \frac{p_0\theta_0}{p_1\theta_1}\label{eq:mu-mu}
		\end{align}
	ここで,圧力\(p\)は定義より以下のように変形できる.
		\begin{equation}
			p = \frac{F}{S} = \frac{mg}{S} = \frac{\rho V}{S}\label{eq:rho}
		\end{equation}
	上式で\(F[\si{N}]\)は流体に作用する力,\(S[\si{m^2}]\)は毛細管の断面積,
	\(m[\si{kg}]\)は流体の質量,\(g[\si{m.s^{-2}}]\)は重力加速度,
	\(\rho[\si{kg.m^{-3}}]\)は流体の密度である.\eqref{eq:rho}式を用いて
	\eqref{eq:mu-mu}式を書き直す.
		\begin{equation}
			\frac{\mu_0}{\mu_1}= \frac{\frac{\rho_0 V}{S}\theta_0}
			{\frac{\rho_1 V}{S}\theta_1}	
		\end{equation}
	断面積\(S\)は同一装置を用いたことによって内半径\(r\)が等しいため同一,重力加速度
	\(g\)についても定数であるため同一である.
	\begin{equation}
		\frac{\mu_0}{\mu_1} = \frac{\rho_0\theta_0}{\rho_1\theta_1}
	\end{equation}
\end{document}
